\documentclass[paper=a4, fontsize=12bp]{scrreprt}
\usepackage[T1]{fontenc}
\usepackage[utf8]{inputenc}
%nachfolgend Paket für Times New Roman
\usepackage{mathptmx}
\usepackage[ngerman]{babel}
%Seitenränder einstellen
\usepackage[head=2cm,bottom=2cm, left=25mm, right= 25mm ]{geometry}

\begin{document}
% Kapitelüberschriften werden mit \chapter{Titel} angegeben
\chapter{Dies wäre der Titel}
%Unterkapitel der ersten Gliederungsebene werden mit \section{Titel} angegeben
\section{Dies wäre also das Unterkapitel}
Lorem ipsum et dolorum...
%Unterkapitel der zweiten Gliederungsebene werden mit \subsection{Titel} angegeben
\subsection{Dies wäre ein Unterkapitel eine Ebene tiefer}
Lorem ipsum et dolorem... oder so ähnlich ;-)
\end{document}