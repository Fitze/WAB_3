%Dies ist die Vorlage für die einzelnen Kapitel, die jeweils mit Chapter als Kapiteltitel starten
\chapter{Kryptographie}
Die Kryptographie ist eine Wissenschaftslehre, die sich mit den Verfahren sowie der Anwendung von Ver- und Entschlüsselung von Informationen befasst. Dabei bedient sie sich mathematischen Hilfsmitteln, um die Daten vor Dritten unzugänglich zu machen. In diesem Kapitel werden zunächst die grundlegende Funktionsweise der Verschlüsselung untersucht, die zum Verständnis der weiteren Unterkapitel notwendig sind. Anschließend werden verschiedene Verfahren beleuchtet, die eine verschlüsselte und geschützte Kommunikation ermöglichen.
%mit \section{title} wird ein Unterkapitel der ersten Gliederungsebene überschrieben
\section{Grundlagen}


%mit \subsection{title} wird ein Unterkapitel der zweiten Gliederungsebene überschrieben
\subsection{Symmetrische Verschlüsselung}

Bei der symmetrischen Verschlüsselung wird sowohl zur Verschlüsselung und Entschlüsselung ein gemeinsamer Schlüssel verwendet. Dieser Schlüssel wird auch Shared Secret genannt. Mithilfe dieses Schlüssels und einem kryptographischen Algorithmus wird die Information des Absenders, auch als Klartext bezeichnet, verschlüsselt. Ein Algorithmus transformiert dabei einen Eingabeparameter in einen Ausgabeparameter. Im Fall eines kryptographischen Algorithmus handelt es sich um eine mathematische Vorschrift, die aus dem Klartext einen sogenannten Geheimtext berechnet. Diesen verschlüsselten Text kann der Empfänger mit dem selben Schlüssel, der zur Verschlüsselung verwendet wurde, entschlüsseln. Die folgende Abbildung zeigt das Funktionsprinzip einer symmetrischen Verschlüsselung
\footnote{Kryptographie, S. 41}
Im elektronischen Briefverkehr wirkt sich die Funktionsweise der symmetrischen Verschlüsselung nachteilig auf deren Anwendung auf. Da beide Kommunikationspartner denselben Schlüssel benötigen, muss dieser zuvor ausgehandelt und übertragen werden. Dieser Umstand stellt ein Risiko bezüglich der Vertraulichkeit dar. Jeder, der über den Schlüssel verfügt, kann auf den Inhalt der Nachricht zugreifen. Dieses Sicherheitsrisiko wird durch die asymmetrische Verschlüsselung beseitigt.

\subsection{Asymmetrische Verschlüsselung}

Wie bei der symmetrischen Verschlüsselung kommt auch bei der asymmetrischen Verschlüsselung, die auch Public-Key-Kryptography genannt wird, ein kryptographischer Algorithmus zum Einsatz. Bei letzterem Verfahren wird statt eines Schlüssels ein Schlüsselpaar verwendet. Dieser besteht aus einem öffentlichen und einem privaten Schlüssel, die mathematisch zusammenhängen. Jeder, der verschlüsselte Nachrichten empfangen möchte, verfügt über ein solches Schlüsselpaar. Der private Schlüssel wird niemals bekanntgegeben, wohingegen der öffentliche Schlüssel jedem zugänglich gemacht werden kann. Obwohl die Schlüssel zusammenhängen, kann aus der Kenntnis des öffentlichen Schlüssels nicht auf den privaten Schlüssel geschlossen werden. Möchte Alice mit diesem Verfahren eine verschlüsselte Mail an Bob versenden, so besorgt sie sich zunächst den öffentlichen Schlüssel von Bob. Damit verschlüsselt sie ihre Nachricht und schickt diese an Bob, der die Nachricht mit seinem privaten Schlüssel entschlüsseln kann. Die Abbildung verdeutlicht noch einmal die Funktionsweise der Public-Key-Verschlüsselung
\footnote{Kryptographie, S. 177}
Mit diesem Verfahren wurde das Sicherheitsrisiko der symmetrischen Verschlüsselung behoben, da der öffentliche Schlüssel zum Verschlüsseln jedem bekannt sein darf. Zur Entschlüsselung wird der dazugehörige private Schlüssel benötigt, der im Besitz des Empfängers ist und niemals veröffentlicht wird.
Ein Sicherheitsrisiko ergibt sich jedoch aus der Tatsache, dass ein Dritter die Übertragung des öffentlichen Schlüssels von Bob abfangen und sich somit als Bob ausgeben kann, indem er seinen öffentlichen Schlüssel publiziert. In diesem sogenannten Man-In-The-Middle-Szenario kann der Dritte nun alle vermeintlich an Bob verschlüsselten Nachrichten lesen, da diese mit seinem öffentlichen Schlüssel verschlüsselt worden sind.
Dies wird durch digitale Signaturen sichergestellt, deren Konzept im nächsten Kapitel näher untersucht wird.

\subsection{Digitale Signaturen}

Digitale Signaturen bieten die Möglichkeit, die menschliche Unterschrift in der digitalen Welt abzubilden. Um dies zu garantieren, müssen folgende Bedingungen eingehalten werden:
\footnote{Kryptographie S. 202}
\begin{itemize}
\item Sie darf nicht zu fälschen sein.
\item Ihre Echtzeit muss überprüfbar sein
\item Sie darf nicht unbemerkt von einem Dokument zum anderen übertragen werden können.
\item Das dazugehörende Dokument darf nicht unbemerkt verändert werden können.
\end{itemize}
Diese Voraussetzungen dienen dazu, die Authentizität sowie die Integrität des Absenders zu gewährleisten. Dazu wird das asymmetrische Verschlüsselungsverfahren genutzt. Dabei verschlüsselt der Absender seine Nachricht mit seinem privaten Schlüssel. Dieser Vorgang wird als signieren bezeichnet. Die resultierende verschlüsselte Nachricht ist die digitale Signatur. Zusammen mit der ursprünglichen Nachricht wird die Signatur zum Empfänger geschickt. Dieser kann nun die Authentizität der Nachricht überprüfen, indem er die Signatur mit dem öffentlichen Schlüssel des Absenders entschlüsselt. Stimmt die entschlüsselte Nachricht mit der originalen Nachricht überein, so kann er sicher sein, dass die Nachricht von dem Absender stammt, da nur mit dessen privatem Schlüssel die Nachricht verschlüsselt sein konnte. Zusätzlich ist damit garantiert, dass die Nachricht vollständig und ungeändert beim Absender angekommen ist.

Beim Signieren wird in der Regel aufgrund des Rechenaufwandes für lange Nachrichten nicht die gesamt Nachricht verschlüsselt, sondern ein sogenannter Hashwert der Nachricht. Hashwerte sind eine Zeichenfolge mit einer bestimmten Länge, die durch eine mathematische Einweg-Hashfunktion generiert werden. Diese Funktionen haben einen Eingabeparameter und berechnen daraus den erwähnten Hashwert. Aus der Kenntis des Hashwertes und der Funktion lässt sich der Eingabeparameter nicht ableiten. Dadurch, dass zu jedem Eingabeparameter nur ein Hashwert existiert, bleibt die Eigenschaft der Integrität erhalten.

\subsection{Zertifikate}
Bei den bisher genannten Verfahren wurde davon ausgegangen, dass der öffentliche Schlüssel wirklich dem Kommunikationspartner gehört. Die Verlässlichkeit des öffentlichen Schlüssels ist durch Szenarien wie dem Man-In-The-Middle-Angriff nicht immer garantiert. Um dies zu erreichen, werden Zertifikate benutzt.
Ein Zertifikat ist ein elektronisches Dokument, das einer Person zugeordnet werden kann. Dieses Dokument enthält neben den persönlichen und weiteren Informationen des Inhabers dessen öffentlichen Schlüssel. Außerdem enthält ein Zertifikat eine Signatur über all den genannten Angaben. Das Signieren wird dabei meist von einer vertrauenswürdigen Instanz durchgeführt, die auch \ac{CA} oder Zertifizierungsstelle genannt wird.
Möchte Alice Bob nun eine verschlüsselte Nachricht schicken, so besorgt sich Alice zunächst Bobs Zertifikat, auf dem sich dessen öffentlicher Schlüssel befindet. Um zu prüfen, ob dieser Schlüssel tatsächlich zu Bob gehört, verifiziert sie nun sein Zertifikat, indem sie die Signatur mit dem öffentlichen Schlüssel der \ac{CA} entschlüsselt. Stimmen beide Schlüssel überein, kann sie davon ausgehen, dass dieser Schlüssel tatsächlich Bob gehört.
Das Sicherheitsrisiko bezüglich der Verlässlichkeit des öffentlichen Schlüssels der Zertifizierungsstelle wird so gelöst, indem ein Zertifikat über den öffentlichen Schlüssel erstellt wird, das von der CA selbst signiert wurde. Dieses Zertifikat wird als self-signed bezeichnet.

\section{Web of Trust}
Das Web of Trust ist ein Vertrauensmodell, bei dem sich die Nutzer gegenseitig vertrauen und somit ein netzartiges Modell entstehen lässt. Die Grundidee ist dabei, dass die Nutzer die öffentlichen Schlüssel voneinander signieren. An einem Beispiel lässt sich das Grundprinzip dieses Modells verdeutlichen \footnote{Angewandte Kryptographie, S. 120}: Carol möchte Bob eine vertrauliche Nachricht schicken. Dazu besorgt er sich Bobs öffentlichen Schlüssel. Zuvor hatte Alice den öffentlichen Schlüssel von Bob signiert. Da Carol Alice vertraut, beschafft sie sich die Signatur von Alice über den öffentlichen Schlüssel von Bob und entschlüsselt diesen mit Alices öffentlichem Schlüssel. Stimmen die Schlüssel überein, so kann Carol den öffentlichen Schlüssel von Bob vertrauen, weil dieser von Alice signiert wurde, der Carol vertraut. Die folgende Abbildung veranschaulicht das Funktionsprinzip:
\footnote{S. 121, Angewandte Kryptographie}.
Bei diesem Modell wird zwischen zwei Arten des Vertrauens unterschieden. Zum einen vertraut man in einen Schlüssel, indem dieser Schlüssel durch jemand anderes, den man vertraut, signiert worden ist. Zum anderen gibt es das Vertrauen in eine Person. Diese beiden Arten sind nicht voneinander abhängig.