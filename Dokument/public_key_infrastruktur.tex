%Dies ist die Vorlage für die einzelnen Kapitel, die jeweils mit Chapter als Kapiteltitel starten
\chapter{Public Key Infrastruktur}

%Problem
- In Kapitel \ref*{chp: zertifikate} Funktionsweise erklärt
- Austausch der Zertifikate zw Kommunikationspartner -> Problem: sie müssen sich "kennen und einen sicheren Weg für Austausch finden" \footcite{BSI}

%Idee
- PKI zur Vereinfachung des Austauschs

%Umsetzung
- Hierarchie von Zertifikaten, die aus einer Wurzelinstanz und dieser untergeordneten Certification Authorities besteht, die wiederum die Zertifikate der Benutzer signieren \footcite[Vgl.][]{Schwenk, S23}
- PCA -> für alle vertrauenswürdig, erstellt ein sog. Wurzel Zertifikat \footcite[Vgl. ][]{ITWissen2012}

- alle weiteren Zertifikate dieser Vertrauenskette werden mit privaten Schlüssel des Wurzelzertifikats signiert, sofern Anforderungen übereinstimmen \ref{chp: Signatur}


%Vorteile
- nicht jedes Zertifikat muss von CA signiert sein
- Es kann auch zwischenzertifikate geben
- die Kette kann sich beliebig lang fortsetzen
- möchte Bob das Zertifikat von Alice prüfen, so schaut er entlang der PKI, solange bis er auf ein vertrauenswürdiges stößt
- möglich, da im Zertifikat die ausstellende Instanz und deren Zertifikat enthalten ist
- Unternehmen: Für MA selbst Zertifikate ausstellen -> weniger Aufwand

- Unter einem Wurzelzertifikat können durch verschiedene CAs verschieden artige Zertifikate ausgestellt werden z.B. für E-Mail, SSL Zertifikate für Webserver, Zertifikate für S/MIME
- Verschiedene Sicherheitspolicies je CA -> Stufe 1 -> gültige E-Mail Adresse; Stufe 3: Personalausweis
- Struktur einer Unternehmung darstellen
\footcite[Vgl.][]{Schwenk, S24}

%Nachteile
- Vertrauenswürdiges CA bekannt machen -> Einbindung in E-Mail Clients
- CA können angegegriffen werden und somit vertrauensunwürdige Zertifikate signiert werden
- Häufigkeit?

%Grenzen
-

%Mailbezug
- Zertifikate beim Versenden von Mails -> eher allg. Part

%mit \section{title} wird ein Unterkapitel der ersten Gliederungsebene überschrieben
%\section{Wichtiges Unterkapitel erster Gliederungsebene}
%Lorem ipsum \dots

%mit \subsection{title} wird ein Unterkapitel der zweiten Gliederungsebene überschrieben
%\subsection{Wichtiges Unterkapitel der zweiten Gliederungsebene}
%Lorem ipsum \dots

%Vom Grundsatz war es das für die Erstellung von Kapiteln, jetzt kommt noch ein kleines Beispiel für Fußnoten
%Dies ist ein großartiges Beispiel \footnote{Wenn einem nix einfällt muss man eben Quatsch schreiben} für eine Fußnote.