\phantomsection \addcontentsline{toc}{chapter}{Abkürzungsverzeichnis}
\renewcommand\refname{Abkürzungsverzeichnis}
\chapter*{Abkürzungsverzeichnis}
\begin{multicols}{2}
\begin{acronym}[TLS/SSL] % In die optionale eckige Klammer die längste Abkürzung schreiben (für gemeinsame Ausrichtung)
	\acro{AES}{Advanced Encryption Standard}
	\acro{CA}{Certificate Authority}
	\acro{DANE}{DNS-based Authentication of Named Entities}
	\acro{DHE}{Diffie-Hellmann-Schlüsselaustausch}
	\acro{DNS}{Domain Name System}
	\acro{DNSSEC}{Domain Name System Security Extensions}
	\acro{ECDH}{Elliptic Curve Diffie-Hellman}
	\acro{EmiG}{E-Mail made in Germany}
	\acro{HTTP}{Hypertext Transfer Protokoll}
	\acro{HTTPS}{Hypertext Transfer Protokoll Secure}
	\acro{ICANN}{Internet Corporation for Assigned Names and Numbers}
	\acro{IP}{Internet Protocol}
	\acro{IPsec}{Internet Protokoll Security}
	\acro{KSK}{Key Sgning Key}
	\acro{MITM}{Man in the Middle}
	\acro{MAC}{Message Authentication Code}
	\acro{OSI}{Open System Interconnection}
	\acro{PGP}{Pretty Good Privacy}
	\acro{PFS}{Perfect Foreward Secrecy}
	\acro{PKI}{Public Key Infrastructure}
	\acro{RSA}{RSA(Rivest, Shamir und Adleman)-Kryptosystem}
	\acro{SMTP}{Simple Mail Transfer Protokoll}
	\acro{SSH}{Secure Shell}
	\acro{SSL}{Secure Socket Layer}
	\acro{S/MIME}{Secure/Multipurpose Internet Mail Extensions}
	\acro{TCP}{Transmission Control Protocol}
	\acro{TLD}{Top-Level-Domain}
	\acro{TLS}{Transport Layer Security}
	\acro{TLSA}{Transport Layer Security Authentication}
	\acro{TLS/SSL}{Transport Layer Security/Secure Socket Layer}
	\acro{TCR}{Trusted Community Representatives}
	\acro{URL}{Unified Ressource Locator}
	\acro{ZSK}{Zone Signing Key}
\end{acronym}
\end{multicols}