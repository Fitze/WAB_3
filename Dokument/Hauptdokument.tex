\documentclass[paper=a4, fontsize=12bp]{scrreprt}
%Einbinden der ben�tigten Pakete �ber die Datei Pakete.tex, wobei die Dateiendung weggelassen wird
\usepackage[T1]{fontenc}
\usepackage[utf8]{inputenc}
\usepackage{mathptmx}				%Font: Times New Roman
%\usepackage[default]{droidserif}	%Font: Droid Serif
\usepackage[ngerman]{babel}
%Seitenränder einstellen
\usepackage[head=2cm,bottom=2cm, left=25mm, right= 25mm ]{geometry}
%Paket für Erzeugung eines Abkürzungsverzeichnis, über das auch per Befehl im späteren Text auf diese verwiesen werden kann
\usepackage[printonlyused]{acronym}
%Paket zur Gestaltung der Kopf- und Fusszeile bei KOMA-Script
\usepackage{scrpage2}
\usepackage[hidelinks]{hyperref}
%Bibliograpie Packages / Settings
\usepackage{ragged2e} % Ermöglicht Flattersatz mit Silbentrennung
\usepackage[babel,german=guillemets]{csquotes}
% damit werden Zitate in französische Anführungszeichen gesetzt
\usepackage[%
backend=biber,% damit wird bestimmt, dass Sie mit der biber.exe arbeiten
bibencoding=utf8,% steht für die Kodierung der .bib-Datei
bibwarn=true,% damit werden ggf. enstandene Fehler ausgegeben
style=alphabetic,% hier wird die DIN 1505-T2 eingebunden
%firstinits=true% der Familienname des Authors wird an die erste Stelle
% gesetzt und anschließend der Vorname nur mit dem ersten
% Buchstaben abgekürzt ausgegeben
]{biblatex}
\usepackage[babel]{microtype} % bringt optischen Randausgleich und
% minimale Skalierung der Buchstaben
\setlength\bibitemsep{8pt} % Abstand zwischen 2 Einträgen im Verzeichnis
% nachfolgend wird der Kopf des Literaturverzeichnisses bestimmt
\defbibheading{online}{\subsection*{Online-Quellen}}
\defbibheading{offline}{\subsection*{Literatur}}
\ExecuteBibliographyOptions{
%isbn=false, % falls die ISBN hinterlegt ist wird diese ausgeblendet
}
\addbibresource{bibliothek.bib} % Einbindung der Literaturquelldatei
%Fortlaufende Fußnoten
\usepackage{chngcntr}
\counterwithout{footnote}{chapter}
%Paket zum Einbinden von Grafiken
\usepackage[pdftex]{graphicx}



% Bibtexkey vor Quellenangabe (Jedoch nur bei stlye=alphabetic)
\DeclareFieldFormat{labelalpha}{\thefield{entrykey}}
\DeclareFieldFormat{extraalpha}{}

\usepackage{filecontents}


%Falls wir Literaturverzeichnis mit BibTEX erstellen wollen m�ssen wir die n�chste Zeile aktivieren und eine entsprechende .bib-Datei erstellen
%\bibliography{Dateiname}

%Mit nachfolgendem Befehl k�nnen LaTeX W�rter �bergeben werden, die es entweder nicht selbstst�ndig trennen kann(und hierdurch die entsprechenden Stellen markiert werden) oder nicht trennen soll
\hyphenation{DANE DNSSEC Fla-schen-hals}

\begin{document}
% \pagestyle{empty} sorgt daf�r, das keine Seitenzahl auf der entsprechenden Seite auftaucht

%entfernt den Einzug nach Abs�tzen
\parindent 0pt

\begin{titlepage}
\begin{figure}
  \begin{center}
    \hbox to \hsize{%
      \begin{tabular}[m]{c}
        \includegraphics[width=2.5cm]{images/HfTL-Logo.png}
      \end{tabular}
      \hfill%
      \begin{tabular}[m]{c}
        Hochschule für Telekommunikation Leipzig (FH)\\
        Wirtschaftsinformatik \\
        Wissenschaftlich angeleitete Berufspraxis III\\
      \end{tabular}%
    }
  \end{center}
\end{figure}

\begin{center}
\rule{0pt}{0pt}
\vfill
\vfill
\vfill
\vfill

\begin{huge}
WAB III Projektarbeit:\\[0.75ex]
\begin{Large}
Analyse technischer Verfahren für sichere\\[0.75ex]
E-Mail-Kommunikation und Bewertung der Implementierung\\[0.75ex]
im Privatanwenderbereich\\[0.75ex]
\end{Large}

\end{huge}

\vfill
\vfill

Projektarbeit\\ von\\

\vspace*{.5cm}
Pascal Feller\\
Daniel Moy\\
Florian Schünhoff\\
Chi Cong Tran\\
\vspace{.5cm}
\today\\

\vfill
\vfill
\vfill
\vfill

\begin{tabular}{rl}
Referent, Betreuer:   & Prof. Dr. - Ing. Undine Pielot\\
\end{tabular}
\end{center}
\end{titlepage}



\newpage
\pagestyle{empty}
\includepdf{anlagen/Selbstaendigkeit.pdf}
%
%
%\text{ }
%\vspace{11.5cm}
%
%
%
%
%Hiermit versichern wir, dass wir die von uns vorgelegte Arbeit selbstständig verfasst haben, dass wir die verwendeten Quellen, Internet-Quellen und Hilfsmittel vollständig angegeben haben und dass wir die Stellen der Arbeit -- einschließlich Tabellen, Karten und Abbildungen~--, die anderen Werken oder dem Internet im Wortlaut oder dem Sinn nach entnommen sind, auf jeden Fall unter Angabe der Quelle als Entlehnung kenntlich gemacht haben.\\
%
%Leipzig, \today\\
%\medskip
%\medskip
%
%(Unterschrift)\bigskip\\
%\underline{~~~~~~~~~~~~~~~~~~~~~~~~~~~~~~~~~~~~~~~~}\\
%Pascal Feller\bigskip\\
%\underline{~~~~~~~~~~~~~~~~~~~~~~~~~~~~~~~~~~~~~~~~}\\
%Daniel Moy\bigskip\\
%\underline{~~~~~~~~~~~~~~~~~~~~~~~~~~~~~~~~~~~~~~~~}\\
%Florian Schünhoff\bigskip\\
%\underline{~~~~~~~~~~~~~~~~~~~~~~~~~~~~~~~~~~~~~~~~}\\
%Chi Cong Tran\\

\newpage


\pagestyle{empty}
%Seitenumbruch
\pagebreak

\pagenumbering{Roman}
%Inhaltsverzeichnis
\tableofcontents
\pagebreak
%Abbildungsverzeichnis
\listoffigures
\pagebreak
%falls ben�tigt auch noch ein Tabellenverzeichnis, m�sste dann die Auskommentierung aufgehoben werden
%\listoftables
%\pagebreak
%einbinden des in Akronyme.tex erstellten Abbildungsverzeichnisses
\phantomsection \addcontentsline{toc}{chapter}{Abkürzungsverzeichnis}
\renewcommand\refname{Abkürzungsverzeichnis} \chapter*{Abkürzungsverzeichnis}
\begin{acronym}[StuRa] % In die optionale eckige Klammer die längste Abkürzung schreiben (für gemeinsame Ausrichtung)
	\acro{DHE}{Diffie-Hellmann-Verfahren}
	\acro{TLS/SSL}{Transport Layer Security/Secure Socket Layer}
	\acro{PFS}{Perfect Foreward Secrecy}
	\acro{DNS}{Domain Name System}
	\acro{DANE}{DNS-based Authentication of Named Entities}
	\acro{AES}{Advanced Encryption Standard}
	\acro{DNSSEC}{Domain Name System Security Extensions}
	\acro{PKI}{Public Key Infrastructure}
	\acro{CA}{Certificate Authority}
	\acro{PGP}{Pretty Good Privacy}
	\acro{S/MIME}{Secure/Multipurpose Internet Mail Extensions}
	\acro{EmiG}{E-Mail made in Germany}
	\acro{TLSA}{Transport Layer Security Authentication}
	\acro{TLS}{Transport Layer Security}
	\acro{SSL}{Secure Socket Layer}
	\acro{HTTP}{Hypertext Transfer Protokoll}
	\acro{OSI}{Open System Interconnection}
	\acro{TCP}{Transmission Control Protocol}
	\acro{MAC}{Message Authentication Code}
\end{acronym}
\pagebreak

%�ndern der Seitennummerierung auf arabische Zahlen und Beginn bei Seite 1
\pagenumbering{arabic}\setcounter{page}{1}
\input{../Vorlagen/Kapitelvorlage}
An dieser Stelle kann dann mal probiert werden ob Flaschenhals jetzt korrekt getrennt wird in dem ganz oft Flaschenhals geschrieben wird: Flaschenhals Flaschenhals Flaschenhals Flaschenhals Flaschenhals Flaschenhals Flaschenhals Flaschenhals Flaschenhals Flaschenhals
%Wenn automatisch erzeugtes Literaturverzeichnis verwendet wird, wird es hier eingef�gt
%\printbibliography
\end{document}