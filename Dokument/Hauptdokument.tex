\documentclass  [paper=a4,
				fontsize=12pt,
				listof=totoc,
				bibliography=totoc
				]{scrreprt}
%Einbinden der benötigten Pakete über die Datei Pakete.tex, wobei die Dateiendung weggelassen wird
\usepackage[T1]{fontenc}
\usepackage[utf8]{inputenc}
\usepackage{mathptmx}				%Font: Times New Roman
%\usepackage[default]{droidserif}	%Font: Droid Serif
\usepackage[ngerman]{babel}
%Seitenränder einstellen
\usepackage[head=2cm,bottom=2cm, left=25mm, right= 25mm ]{geometry}
%Paket für Erzeugung eines Abkürzungsverzeichnis, über das auch per Befehl im späteren Text auf diese verwiesen werden kann
\usepackage[printonlyused]{acronym}
%Paket zur Gestaltung der Kopf- und Fusszeile bei KOMA-Script
\usepackage{scrpage2}
\usepackage[hidelinks]{hyperref}
%Bibliograpie Packages / Settings
\usepackage{ragged2e} % Ermöglicht Flattersatz mit Silbentrennung
\usepackage[babel,german=guillemets]{csquotes}
% damit werden Zitate in französische Anführungszeichen gesetzt
\usepackage[%
backend=biber,% damit wird bestimmt, dass Sie mit der biber.exe arbeiten
bibencoding=utf8,% steht für die Kodierung der .bib-Datei
bibwarn=true,% damit werden ggf. enstandene Fehler ausgegeben
style=alphabetic,% hier wird die DIN 1505-T2 eingebunden
%firstinits=true% der Familienname des Authors wird an die erste Stelle
% gesetzt und anschließend der Vorname nur mit dem ersten
% Buchstaben abgekürzt ausgegeben
]{biblatex}
\usepackage[babel]{microtype} % bringt optischen Randausgleich und
% minimale Skalierung der Buchstaben
\setlength\bibitemsep{8pt} % Abstand zwischen 2 Einträgen im Verzeichnis
% nachfolgend wird der Kopf des Literaturverzeichnisses bestimmt
\defbibheading{online}{\subsection*{Online-Quellen}}
\defbibheading{offline}{\subsection*{Literatur}}
\ExecuteBibliographyOptions{
%isbn=false, % falls die ISBN hinterlegt ist wird diese ausgeblendet
}
\addbibresource{bibliothek.bib} % Einbindung der Literaturquelldatei
%Fortlaufende Fußnoten
\usepackage{chngcntr}
\counterwithout{footnote}{chapter}
%Paket zum Einbinden von Grafiken
\usepackage[pdftex]{graphicx}



% Bibtexkey vor Quellenangabe (Jedoch nur bei stlye=alphabetic)
\DeclareFieldFormat{labelalpha}{\thefield{entrykey}}
\DeclareFieldFormat{extraalpha}{}

\usepackage{filecontents}


%Falls wir Literaturverzeichnis mit BibTEX erstellen wollen müssen wir die nächste Zeile aktivieren und eine entsprechende .bib-Datei erstellen
%\bibliography{bibliothek} 

%Mit nachfolgendem Befehl können LaTeX Wörter übergeben werden, die es entweder nicht selbstständig trennen kann(und hierdurch die entsprechenden Stellen markiert werden) oder nicht trennen soll
\hyphenation{golem Trans-port-ver-schlüs-sel-ung TLS-Transport-verschlüsselung DNSSEC}

\begin{document}
% \pagestyle{empty} sorgt dafür, das keine Seitenzahl auf der entsprechenden Seite auftaucht

%entfernt den Einzug nach Absätzen
\parindent 0pt
%\begin{titlepage}
\begin{figure}
  \begin{center}
    \hbox to \hsize{%
      \begin{tabular}[m]{c}
        \includegraphics[width=2.5cm]{images/HfTL-Logo.png}
      \end{tabular}
      \hfill%
      \begin{tabular}[m]{c}
        Hochschule für Telekommunikation Leipzig (FH)\\
        Wirtschaftsinformatik \\
        Wissenschaftlich angeleitete Berufspraxis III\\
      \end{tabular}%
    }
  \end{center}
\end{figure}

\begin{center}
\rule{0pt}{0pt}
\vfill
\vfill
\vfill
\vfill

\begin{huge}
WAB III Projektarbeit:\\[0.75ex]
\begin{Large}
Analyse technischer Verfahren für sichere\\[0.75ex]
E-Mail-Kommunikation und Bewertung der Implementierung\\[0.75ex]
im Privatanwenderbereich\\[0.75ex]
\end{Large}

\end{huge}

\vfill
\vfill

Projektarbeit\\ von\\

\vspace*{.5cm}
Pascal Feller\\
Daniel Moy\\
Florian Schünhoff\\
Chi Cong Tran\\
\vspace{.5cm}
\today\\

\vfill
\vfill
\vfill
\vfill

\begin{tabular}{rl}
Referent, Betreuer:   & Prof. Dr. - Ing. Undine Pielot\\
\end{tabular}
\end{center}
\end{titlepage}



\newpage
\pagestyle{empty}
\includepdf{anlagen/Selbstaendigkeit.pdf}
%
%
%\text{ }
%\vspace{11.5cm}
%
%
%
%
%Hiermit versichern wir, dass wir die von uns vorgelegte Arbeit selbstständig verfasst haben, dass wir die verwendeten Quellen, Internet-Quellen und Hilfsmittel vollständig angegeben haben und dass wir die Stellen der Arbeit -- einschließlich Tabellen, Karten und Abbildungen~--, die anderen Werken oder dem Internet im Wortlaut oder dem Sinn nach entnommen sind, auf jeden Fall unter Angabe der Quelle als Entlehnung kenntlich gemacht haben.\\
%
%Leipzig, \today\\
%\medskip
%\medskip
%
%(Unterschrift)\bigskip\\
%\underline{~~~~~~~~~~~~~~~~~~~~~~~~~~~~~~~~~~~~~~~~}\\
%Pascal Feller\bigskip\\
%\underline{~~~~~~~~~~~~~~~~~~~~~~~~~~~~~~~~~~~~~~~~}\\
%Daniel Moy\bigskip\\
%\underline{~~~~~~~~~~~~~~~~~~~~~~~~~~~~~~~~~~~~~~~~}\\
%Florian Schünhoff\bigskip\\
%\underline{~~~~~~~~~~~~~~~~~~~~~~~~~~~~~~~~~~~~~~~~}\\
%Chi Cong Tran\\

\newpage


\pagestyle{empty}
%Seitenumbruch
\pagebreak

\pagenumbering{Roman}
%Inhaltsverzeichnis
\tableofcontents
\pagebreak
%Abbildungsverzeichnis
\listoffigures
\pagebreak
%falls benötigt auch noch ein Tabellenverzeichnis, müsste dann die Auskommentierung aufgehoben werden
%\listoftables
%\pagebreak
%einbinden des in Akronyme.tex erstellten Abbildungsverzeichnisses
\phantomsection \addcontentsline{toc}{chapter}{Abkürzungsverzeichnis}
\renewcommand\refname{Abkürzungsverzeichnis} \chapter*{Abkürzungsverzeichnis}
\begin{acronym}[StuRa] % In die optionale eckige Klammer die längste Abkürzung schreiben (für gemeinsame Ausrichtung)
	\acro{DHE}{Diffie-Hellmann-Verfahren}
	\acro{TLS/SSL}{Transport Layer Security/Secure Socket Layer}
	\acro{PFS}{Perfect Foreward Secrecy}
	\acro{DNS}{Domain Name System}
	\acro{DANE}{DNS-based Authentication of Named Entities}
	\acro{AES}{Advanced Encryption Standard}
	\acro{DNSSEC}{Domain Name System Security Extensions}
	\acro{PKI}{Public Key Infrastructure}
	\acro{CA}{Certificate Authority}
	\acro{PGP}{Pretty Good Privacy}
	\acro{S/MIME}{Secure/Multipurpose Internet Mail Extensions}
	\acro{EmiG}{E-Mail made in Germany}
	\acro{TLSA}{Transport Layer Security Authentication}
	\acro{TLS}{Transport Layer Security}
	\acro{SSL}{Secure Socket Layer}
	\acro{HTTP}{Hypertext Transfer Protokoll}
	\acro{OSI}{Open System Interconnection}
	\acro{TCP}{Transmission Control Protocol}
	\acro{MAC}{Message Authentication Code}
\end{acronym}
\pagebreak

%ändern der Seitennummerierung auf arabische Zahlen und Beginn bei Seite 1
\pagenumbering{arabic}\setcounter{page}{1}
%\input{../Vorlagen/Kapitelvorlage}
%%Dies ist die Vorlage für die einzelnen Kapitel, die jeweils mit Chapter als Kapiteltitel starten
\chapter{Datensicherheit 1x1}
Unter Datensicherheit wird der Schutz von Daten in den Aspekten Verfügbarkeit, Vertraulichkeit und Integrität verstanden. \footcite{BSI2014} Im Gegensatz dazu beschreibt Datenschutz den Schutz von und den vertrauensvollen Umgang mit persönlichen Daten. In der IT-Sicherheit wird zusätzlich des Aspekt der Authentizität berücksichtigt \footcite{Berliner2014}. In diesem Abschnitt werden diese Aspekte näher betrachtet, da diese für das Verständnis der vorgestellten Techniken in den späteren Kapitel notwendig sind.

%mit \section{title} wird ein Unterkapitel der ersten Gliederungsebene überschrieben
\section{Verfügbarkeit}

Unter Verfügbarkeit wird das Vorhandensein von Infrastruktur, Software, sämtliche IT-Dienstleistungen sowie -Funktionalitäten und Daten verstanden, so dass die Anwender bei Bedarf darauf zugreifen und nutzen können. Um dies zu gewährleisten, muss verhindert werden, dass
\begin{itemize}
\item Daten verschwinden oder nicht zugreifbar sind, wenn sie gebraucht werden,
\item Programme nicht funktionsbereit sind, wenn sie aufgerufen werden sollen,
\item Hardware und sonstige notwendige Mittel nicht funktionsfähig oder gar verschwunden sind, wenn sie für die Verarbeitung benötigt wird. \footcite{Berliner2014}
\end{itemize}

\section{Integrität}
Unter der Integrität der Daten wird verstanden, dass die Daten bei der Übertragung nicht unbemerkt verändert werden können und somit vollständig übermittelt worden sind. Bei diesem Aspekt geht es demzufolge um die Unversehrtheit der Nachricht.

\section{Vertraulichkeit}

Unter Vertraulichkeit versteht man den Schutz der Nachricht vor unbefugtem Zugriff durch Dritte. Nur der gewünschte Empfänger soll in der Lage sein, den Inhalt der Nachricht zu erfahren. Dazu werden mathematische Verfahren genutzt, die in den späteren Kapiteln behandelt werden.

\section{Authentizität}

Bei der Authentizität geht es um den Nachweis, dass die beteiligten Kommunikationspartner tatsächlich diejenigen sind, für die sie sich ausgeben.
%Dies ist die Vorlage für die einzelnen Kapitel, die jeweils mit Chapter als Kapiteltitel starten
\chapter{Sicherheitsniveaus}

Das nachfolgende Kapitel geht auf verschiedene Sicherheitsbedürfnisse eines Nutzers ein. Gerade im Hinblick auf den Grad der Verschlüsselung bei der E-Mail Kommunikation ist es wichtig sich darüber bewusst zu werden, wie sensibel die Information ist, die man versenden möchte. Denn jede Verschlüsselung ist mit einem bestimmten Aufwand verbunden und folglich ist abzuwägen, welcher Verschlüsselungsaufwand dem Nutzer die zu versendende Information wert ist. 
Beispielsweise ist aus Sicht der Autoren der potentielle Schaden gering, wenn eine E-Card zu den Weihnachtsfeiertagen an den nicht rechtmäßigen Empfänger gerät, sodass für die Versendung einer solchen Information der Grad der Verschlüsselung niedrig und somit der Aufwand niedrig ausfällt. Dahingegen ist der potentielle Schaden größer, wenn es sich bei der versendeten Information um beispielsweise die eigenen Kontodaten handelt, was wiederum bedeutet, dass der Nutzer bereit ist einen höheren Aufwand zu betreiben, um diese Inhalte auf eine sichere Art und Weise via E-Mail zu übermitteln.


Das Sicherheitsbedürfnis einer jeden Person kann unterschiedlich stark ausgeprägt sein. Daher ist es den Autoren nicht möglich, eine allgemein gültige Auflistung aller möglichen Szenarien der E-Mail Kommunikation bereitzustellen, aus welcher die Teilnehmer der Zielgruppe lediglich das richtige Szenario heraussuchen müssen und dadurch den optimalen Grad der Verschlüsselung erhalten. Stattdessen wird in Abbildung \ref{img:sicherheitsniveaus} eine Übersicht präsentiert, die es dem Nutzer erlaubt auf Basis seines eigenen Sicherheitsbedürfnisses und mit Hilfe festgelegter Kriterien für die Übermittlung einer ganz bestimmten Information ein geeignetes Sicherheitsniveau zu ermitteln. In den nachfolgenden dieser Arbeit werden verschiedene Möglichkeiten der Verschlüsselung von E-Mail Kommunikation vorgestellt und jeweils passenden Sicherheitsniveaus zugeordnet. Dabei erfolgt diese Zuordnung mit dem Ziel, einen optimalen Ausgleich zwischen Notwendigkeit und Aufwand der Verschlüsselung von E-Mails zu erhalten, sodass der Nutzer nach der Ermittlung eines geeigneten Sicherheitsniveaus eine aus Sicht der Autoren geeignete Verschlüsselungsmethode ermitteln kann.

\pagebreak


\begin{figure}[h] %{wrapfigure}[36]{l}{1.0\textwidth}
	%\begin{center}
\includegraphics[width=16cm]{images/sicherheitsniveaus.jpg}
	%\end{center}

\caption {Sicherheitsniveaus} %Bildunterschrift, erstes Argument ist für Abbildungsverzeichnis ohne Fußnote; \footnotemark ist ein Platzhalter für die Fußnote
\label{img:sicherheitsniveaus} %für Bezüge auf diese Abbildung

\end{figure} %wrapfigure}


Die Abbildung \ref{img:sicherheitsniveaus} stellt im Tabellenkopf vier verschiedene Sicherheitsniveaus dar: \textit{Streng Vertraulich, Vertraulich, Privat und Öffentlich}. Dabei nimmt das Sicherheitsbedürfnis sowie der Verschlüsselungsaufwand von Streng Vertraulich hin zu Öffentlich ab.
In der ersten Spalte sind verschiedene Merkmale aufgelistet, welche es dem Nutzer ermöglichen sollen, für eine ganz bestimmte Information ein geeignetes Sicherheitsniveau zu bestimmen.
Alle weiteren Felder enthalten die Ausprägung des Merkmals innerhalb des jeweiligen Sicherheitsniveaus.

Der Wert der Information und deren Schutzwürdigkeit beschreiben, wie wertvoll die zu versendende E-Mail für einen nicht rechtmäßigen Empfänger ist und welche Notwendigkeit des Schutzes daraus folgt.
Der potentielle Schaden stellt das Ausmaß dar, welches eintritt für den Fall, dass die E-Mail durch einen unberechtigten Dritte gelesen wird. Dieser Schaden kann verschiedener Art sein. Zum Beispiel können daraus rechtliche Konsequenzen erfolgen, es kann zu einem finanziellen Verlust führen oder mit einer Schädigung des Images des Senders einhergehen \footcite[Vgl.][]{Reinhausen GmbH, S. 6}. Aus dem potentiellen Schaden lässt sich außerdem gut der Einfluss auf die eigene Privatsphäre ableiten.
Das Merkmal \textit{Personenbezogene Daten} besagt, dass die zu versendende Information personenbezogene Daten enthält und daher grundsätzlich das Sicherheitsniveau \textit{Vertraulich} zu wählen ist\footcite[Vgl.][]{TSE}. In Abhängigkeit von der Ausprägung der anderen Merkmale, kann als resultierendes Sicherheitsniveau auch \textit{Streng Vertraulich} oder \textit{Privat} ermittelt werden.
Der Autorisierte Personenkreis ist eine weitere Eigenschaft anhand derer der Nutzer ein passendes Sicherheitsniveau bestimmen kann. Grundsätzlich gilt: je wertvoller die Information, desto geringer ist der Personenkreis, der Einblick in die zu versendende E-Mail erhalten darf \footcite[Vgl.][]{TSE}. Daraus folgt, dass eine streng vertrauliche Nachricht ausschließlich zwischen dem Sender und dem Empfänger ausgetauscht wird und in der Regel keine weitere Person über den Inhalt erfahren darf. Eine Ausnahme an dieser Stelle sind allenfalls engste Verwandte. Dahingegen ist für eine Nachricht, deren Inhalt prinzipiell jedermann erfahren darf, das Sicherheitsniveau \textit{Öffentlich} zu wählen\footcite[Vgl.][]{Reinhausen GmbH, S. 10}.
Die \textit{Auswirkung bei Integritätsverletzung} beschreibt das eingetretene Ausmaß, wenn die E-Mail in die Hände eines Angreifers gelangt ist.
Eine weitere Möglichkeit ein geeignetes Sicherheitsniveau zu ermitteln ist die Überlegung, wie der Inhalt der E-Mail auf dem Postweg versandt werden würde. Für eine streng vertrauliche Information würde ein Einschreiben gewählt werden oder gänzlich auf den Postweg verzichtet und stattdessen die Nachricht persönlich überbracht werden. Dahingegen ist für Information, die ohne Bedenken  auf einer Postkarte übermittelt werden können, das Sicherheitsniveau \textit{Öffentlich} zu wählen 
Das letzte Merkmal beschreibt das Aufkommen der einzelnen Sicherheitsniveaus. Streng vertrauliche Informationen sind sehr selten und am häufigsten werden öffentliche Nachrichten ausgetauscht \footcite[Vgl.][]{TSE}.

Anhand des nachfolgenden Beispiels soll der Umgang mit der Abbildung \ref{img:sicherheitsniveaus} verdeutlicht werden. Hierbei ist zu erwähnen, dass das resultierende Sicherheitsniveau auf Basis des beschriebenen Sicherheitsbedürfnisses ermittelt wurde und für die gleiche Information bei anderen Nutzern unterschiedlich ausfallen kann:

Herr Meier war vor kurzem in einen Auffahrunfall verwickelt, den ein unachtsamer Autofahrer verursacht hatte. Daraufhin brachte Herr Meier sein Fahrzeug in die Werkstatt und ließ ein Gutachten des Schadens erstellen. Dieses Gutachten möchte er nun zusammen mit seinen Kontodaten via E-Mail an die Versicherung des Unfallverursachers senden. 
Der Wert dieser Information ist hoch, denn einerseits sind die Kontodaten in der Nachricht vorhanden. Andererseits ist in dem Gutachten Herr Meiers Anschrift angegeben und es lässt sich aus den Fotos und der Schadenshöhe des Gutachtens ableiten, dass Herr Meier einen Luxuswagen besitzt. Daraus wiederum lassen Rückschlüsse auf seine finanzielle Situation schließen. Der potentielle Schaden, der sich daraus ergibt, ist hoch bis äußerst hoch. Denn bei ausreichender krimineller Energie können nicht nur die Kontodaten missbraucht werden, sondern mit Hilfe der Anschrift kann der Wohnsitz von Herrn Meier ausgekundschaftet und beispielsweise bei seiner Abwesenheit in sein Anwesen eingebrochen werden. 
Die Anschrift stellt personenbezogene Daten dar und ermöglicht einen hohen Einfluss auf die Privatsphäre von Herrn Meier.
Der Autorisierte Personenkreis für diese Nachricht beschränkt sich auf den Sender (Herr Meier), den Empfänger (die Versicherung) sowie auf seine Frau und auf die Werkstatt, die das Gutachten erstellt hat. Seinen Freunden und Kollegen hat Herr Meier zwar auch von dem Unfall erzählt. Allerdings wissen diese keine genaueren Details hinsichtlich des Schadens und dessen Höhe.
Hätte die Versicherung den E-Mail Service nicht angegeben, so würde Herr Meier die Unterlagen des Gutachtens mit einem Standard-Brief versenden. Die Kontodaten würde er mittels Einschreiben verschicken.

Somit kann als Resultat für dieses Beispiel unter dem beschriebenen Sicherheitsbedürfnis ein Sicherheitsniveau von \textit{Vertraulich} ermittelt werden. Welches Verschlüsselungsverfahren konkret für dieses Sicherheitsniveau anzuwenden ist, wird in den nachfolgenden Kapiteln beschrieben.


%Dies ist die Vorlage für die einzelnen Kapitel, die jeweils mit Chapter als Kapiteltitel starten
\chapter{Kryptographie}
Die Kryptographie ist eine Wissenschaftslehre, die sich mit den Verfahren der Ver- und Entschlüsselung von Informationen sowie deren Anwendung befasst. Meistens werden dazu geheime Schlüssel oder Schlüsselpaare verwendet. Damit lassen sich mithilfe mathematischen Berechnungsverfahren, sogenannten Algorithmen, Nachrichten verschlüsseln. Heutige Verschlüsselungsverfahren basieren entweder auf einem symmetrischen oder asymmetrischen Algorithmus.

%mit \section{title} wird ein Unterkapitel der ersten Gliederungsebene überschrieben
\section{Grundlagen}


%mit \subsection{title} wird ein Unterkapitel der zweiten Gliederungsebene überschrieben
\subsection{Symmetrische Verschlüsselung}

Bei der symmetrischen Verschlüsselung wird sowohl zur Ver- und Entschlüsselung derselbe Schlüssel verwendet, der auch Shared Secret genannt wird. Mithilfe dieses Schlüssels und einem symmetrischen Verschlüsselungsalgorithmus wird die Nachricht des Absenders verschlüsselt. In diesem Zusammenhang wird der "lesbare Text einer Nachricht [...] Klartext [..] genannt" \footcite[S. 21]{Ertel2012}. Aus dem Shared Secret und dem Klartext wird durch eine mathematische Vorschrift ein Geheimtext erzeugt. Dieser verschlüsselte Text kann ausschließlich mit dem gleichen Schlüssel entschlüsselt werden, mit dem er verschlüsselt wurde.
Diese Tatsache wirkt sich im elektronischen Briefverkehr nachteilig auf die Anwendung der symmetrischen Verschlüsselung aus. Da beide Kommunikationspartner denselben Schlüssel benötigen, muss dieser zuvor ausgehandelt und übertragen werden. Die Übertragung dieses Schlüssels stellt ein Sicherheitsrisiko dar. Wird die Übertragung der Schlüssels abgehört, können Dritte mit seiner Hilfe die verschlüsselten Nachrichten mitlesen.

\subsection{Asymmetrische Verschlüsselung}

Wie bei der symmetrischen Verschlüsselung kommt auch bei der asymmetrischen Verschlüsselung, die auch Public-Key-Kryptography genannt wird, ein kryptographischer Algorithmus zum Einsatz. Hierbei wird jedoch wird statt eines gemeinsamen Schlüssels ein Schlüsselpaar verwendet. Dieser besteht aus einem öffentlichen und einem privaten Schlüssel, die mathematisch zusammenhängen. Jeder, der verschlüsselte Nachrichten empfangen möchte, verfügt über ein solches Schlüsselpaar. Der private Schlüssel wird niemals bekanntgegeben, wohingegen der öffentliche Schlüssel jedem zugänglich gemacht werden kann. Obwohl die Schlüsseln zusammenhängen, kann aus der Kenntnis des öffentlichen Schlüssels nicht auf den privaten Schlüssel geschlossen werden \footcite[S. 177]{Schmeh2013}.
Soll eine Nachricht durch ein asymmetrisches Verschlüsselungsverfahren verschlüsselt verschickt werden, wird zunächst der öffentliche Schlüssel des Empfängers benötigt. Zusammen mit der Nachricht wird der Geheimtext erzeugt und an den Empfänger geschickt. Zum Entschlüsseln der Nachricht wird der private Schlüssel des Empfängers verwendet, der zum bei der Verschlüsselung genutztem öffentlichen Schlüssel gehört.

Mit diesem Verfahren wurde das Sicherheitsrisiko der symmetrischen Verschlüsselung gelöst, da der öffentliche Schlüssel zum Verschlüsseln jedem bekannt sein darf. Zur Entschlüsselung wird der dazugehörige private Schlüssel benötigt, der im Besitz des Empfängers ist und niemals veröffentlicht wird.
Ein Sicherheitsrisiko ergibt sich jedoch aus der Tatsache, dass ein Dritter die Übertragung des öffentlichen Schlüssels abfangen kann und dem Absender stattdessen seinen öffentlichen Schlüssel überträgt. Damit ist er in der Lage, alle vom Absender verschlüsselten Nachrichten ohne dessen Kenntnis zu entschlüsseln. Daher muss "bei der Verwendung eines fremden Schlüssels [..] möglichst immer die Authentizität des Schlüssels geprüft bzw. sichergestellt werden." \footcite[S. 90]{Ertel2012}

Zwei Aspekte sind im Zusammenhang mit der asymmetrischen Verschlüsselung erwähnenswert. Der im Jahr 1978 erfundene asymmetrische RSA-Algorithmus, der nach seinen Erfindern R. Rivest, A Shamir und L Adleman benannt, hebt sich vor allem durch seine Einfachheit hervor \footcite[S. 79]{Ertel2012} und der Diffie-Hellman-Schlüsselaustausch. Mit diesem Protokoll, das Eigenschaften von asymmetrischer Verschlüsselung aufweist, können geheime Schlüssel problemlos über einen abgehörten Kanal übertragen werden. Nach Ablauf der Vereinbarung kennen nur beide Kommunikationspartner den vereinbarten Schlüssel \footcite[S. 129]{Stephan2011}

\subsection{Digitale Signaturen}

Asymmetrische Verschlüsselungsverfahren ermöglichen, die menschliche Unterschrift in der digitalen Welt abzubilden. Diese Funktionalität wird mit digitalen Signaturen umgesetzt. Damit eine digitale Signatur den Anforderungen einer menschlichen Unterschrift erfüllt, müssen einige Bedingungen eingehalten werden:
\footcite[S. 202]{Schmeh2013}
\begin{itemize}
\item Sie darf nicht zu fälschen sein.
\item Ihre Echtzeit muss überprüfbar sein
\item Sie darf nicht unbemerkt von einem Dokument zum anderen übertragen werden können.
\item Das dazugehörende Dokument darf nicht unbemerkt verändert werden können.
\end{itemize}
Diese Voraussetzungen dienen dazu, die Verbindlichkeit des Absenders sowie die Integrität der Nachricht zu gewährleisten. Beim Signieren verschlüsselt der Absender seine Nachricht mit seinem privaten Schlüssel. Die resultierende Nachricht ist die digitale Signatur. In der Regel wird
beim Signieren aufgrund des Rechenaufwandes für lange Nachrichten nicht die gesamte Nachricht verschlüsselt, sondern ein sogenannter Hashwert. Hashwerte sind eine Zeichenfolge mit einer bestimmten Länge, die durch eine mathematische Einweg-Hashfunktion generiert werden. Diese Funktionen haben einen Eingabeparameter und berechnen daraus einen Hashwert. Aus der Kenntnis des Hashwertes oder der Hashfunktion lässt sich der Eingabeparameter nicht ableiten, wodurch die Integrität einer signierten Nachricht gewährleistet ist.
Die signierte Nachricht wird im nächsten Schritt an den Empfänger geschickt. Dieser kann nun die Verbindlichkeit der Nachricht überprüfen, indem er die Signatur mit dem öffentlichen Schlüssel des Absenders entschlüsselt. Dazu berechnet er den Hashwert der erhaltenen Nachricht und vergleicht diesen mit dem zuvor entschlüsselten Hashwert des Absenders. Diese Überprüfung wird dabei auch Verifizierung genannt. Stimmen beide überein, kann er sicher sein, dass die Nachricht von dem Absender stammt, da nur mit dessen privatem Schlüssel die Nachricht verschlüsselt werden konnte. Zusätzlich ist damit garantiert, dass die Nachricht vollständig und ungeändert beim Absender angekommen ist.

Durch die Anwendung des asymmetrischen Verschlüsselungsverfahrens beim Signieren bleibt die Authentizität der öffentlichen Schlüssels weiterhin ein Sicherheitsrisiko. Das Risiko besteht darin, dass	"man einem öffentlichen Schlüssel nicht ansieht, wem er gehört" \footcite[S. 506]{Schmeh2013}.

\subsection{Zertifikate}
Bei den bisher genannten Verfahren wurde davon ausgegangen, dass der öffentliche Schlüssel wirklich dem beabsichtigten Kommunikationspartner gehört. Die Authentizität des öffentlichen Schlüssels ist durch Szenarien wie dem Man-In-The-Middle-Angriff nicht immer garantiert. Um die Authentizität des öffentlichen Schlüssels sicherzustellen, werden Zertifikate verwendet.

Ein Zertifikat ist ein elektronisches Dokument, das einer Person zugeordnet werden kann. Dieses Dokument enthält neben den persönlichen und weiteren Informationen des Inhabers dessen öffentlichen Schlüssel. Außerdem enthält ein Zertifikat eine Signatur über all den genannten Angaben. Das Signieren wird dabei von einer vertrauenswürdigen Instanz durchgeführt, die auch \ac{CA} oder Zertifizierungsstelle genannt wird.

Zum Versenden einer verschlüsselten Nachricht wird zunächst das Zertifikat vom Empfänger besorgt. Der Absender überprüft die Authentizität des öffentlichen Schlüssels des Empfängers, indem er die Signatur unter Verwendung des öffentlichen Schlüsses des \ac{CA}s] verifiziert. Mit der Verifizierung ist gewährleistet, dass der öffentliche Schlüssel auf dem Zertifikat dem Zertifikatsinhaber gehört.

Das Sicherheitsrisiko bezüglich der Authentizität des öffentlichen Schlüssels der Zertifizierungsstelle wird gelöst, indem ein Zertifikat über den seinen öffentlichen Schlüssel erstellt wird, das von der \ac{CA} selbst signiert wurde. Dieses Zertifikat wird als self-signed bezeichnet.

\section{Web of Trust}
Das Web of Trust ist ein Vertrauensmodell, bei dem sich die Nutzer gegenseitig vertrauen und somit ein netzartiges Modell entstehen lässt. Die Grundidee ist, dass die Nutzer dieses Modells gegenseitig ihre öffentlichen Schlüsseln signieren. Im Gegensatz zum hierarchischen Verfahren gibt es keine zentrale Zertifizierungsstelle.
Die Funktionsweise des Web of Trust wird anhand eines Beispiels erläutert.:
	\begin{center}
		\includegraphics[width=0.4\textwidth]{images/WOT.png}
	\end{center}
	\caption[Web of Trust Vertrauensmodell]{Web of Trust Vertrauensmodell\footnotemark}
	In Anlehnung an: \footcite[S. 120]{Ertel2012}
Teilnehmer C möchte Teilnehmer B eine verschlüsselte Nachricht schicken. Dazu besorgt er sich zunächst das Zertifikat von Teilnehmer B. Dieser wurde zuvor von Teilnehmer A signiert. Da Teilnehmer C Teilnehmer A vertraut, beschafft sich Teilnehmer C den öffentlichen Schlüssel von Teilnehmer A und verifiziert mit damit das Zertifikat von Teilnehmer B. Ist die Verifizierung erfolgreich, so kann Teilnehmer C den öffentlichen Schlüssel von Teilnehmer B vertrauen.

Bei diesem Modell wird zwischen zwei Arten des Vertrauens unterschieden. Einerseits existiert das Vertrauen in eine Person bzw. dessen Signatur. Andererseits besteht ein Vertrauen in einen signierten Schlüssel eines Dritten, der von einer vertrauensvollen Person signiert wurde. Beide Arten des Vertrauens können unabhängig voneinander existieren.
\chapter{Schlüsselaustausch}
%mit \section{title} wird ein Unterkapitel der ersten Gliederungsebene überschrieben
\section{Perfect Forward Secrecy - PFS}
%Problem
Beim klassischen Schlüsselaustausch werden die Sitzungsschlüssel durch den Public Key innerhalb des Server-Zertifikat übertragen.\footnote{Golem} Dies geschieht mittels \ac{RSA}-Verfahren. Verschlüsselte Kommunikation ist jedoch nur solang die Schlüssel geheim bleiben sicher. Die Gefahr beim klassischen \ac{RSA}-\ac{PKI}-Verfahren ist dass vergangene Kommunikation nachträglich zu jedem Zeitpunkt entschlüsselt werden kann, sobald Angreifer in Besitz des Private Keys sind.\medskip\\
%Idee
Sinnvoller ist es die Sitzungsschlüssel %REFERENZ Erklären was Sitzungsschlüssel sind
zum Einen nicht mehr zu übertragen und zum Anderen unabhängig voneinander ständig neu zu generieren und bei Terminierung zu löschen. Realisiert wird dies durch die Protokoll-Eigenschaft Forward Secrecy, die im Kryptographischen Fachjargon auch \ac{PFS}\footcite[test]{Boeck2013} genannt wird. \medskip\\
%Umsetzung
Das \ac{DHE} ermöglicht die Aushandlung eines Sitzungsschlüssels bei dem die Kommunikationspartner verschiedene Nachrichten senden und sich auf einen Sitzungsschlüssel einigen können, ohne diesen je übertragen zu haben. Dieser Schlüssel ist auch nur für die aktuelle Verbindung gültig und wird anschließend gelöscht. Der Public-Key des Servers wird weiterhin übertragen, jedoch nur um den Schlüsselaustausch zu signieren. Die Verschlüsselungsverfahren \ac{TLS/SSL} und IPsec beherrschen bereits \ac{PFS}.\medskip\\
%Vorteile
Aufgezeichnete verschlüsselte Daten können somit bei Besitz des privaten Schlüssels nicht entschlüsselt werden. Zudem wird einfaches Belauschen einer aktiven Verbindung deutlich erschwert, denn es müsste die gesamte Kommunikation mit einem gezieltem \ac{MITM}-Angriff manipuliert werden. Für diese Problematik gibt es wiederum moderne Ansätze wie \ac{DANE}, die in Kombination mit \ac{PFS} aktuell bei der Verschlüsselung von Verbindungen höchsten Sicherheitsansprüchen entsprechen, indem zusätzlich die Authentizät der Kommunikation gewährleistet wird.\medskip\\ %REFERENZ MitM erklären 
%Nachteile
Nachteile gibt es lediglich bei der Verwendung des bereits überholten, und seit Jahren als geknackt bekannte \ac{DHE}-Verfahren, denn dabei verzögert sich zusätzlich der Verbindungaufbau. Die Moduluslänge der Schlüssel ist Minimum 1024 Bit, und längere Schlüssel mit 2048 oder 4096 Bit sind debi nicht sicher. Der moderne Nachfolger mit elliptischen Kurven \ac{ECDH} gilt aktuell als sicher und benötigt dabei weniger als 1024 Bit und verzögert den Verbindungsaufbau nur unweigerlich.\medskip\\
%Grenzen
Obwohl es Forward Secrecy bereits seit 1999 im \ac{TLS} Standard 1.0 \footnote{golem} vorgesehen ist und somit essenzieller Bestandteil von Verschlüsselung ist, hat sich \ac{PFS} noch nicht als Standard durchgesetzt.\footnote{https://www.trustworthyinternet.org/ssl-pulse/} Dies liegt zum einen an den Webservern. Mit einem Apache Webserver ist nur eine Moduluslänge von 1024 Bit vorgesehen. Beim Einsatz von \ac{DHE} würden Provider damit ihre Server daher unsicher betreiben. Zum Anderen sind es auf Client-Seite die Browser die noch nicht mitspielen. Der Internet Explorer verschlüsselt nur nach DSS, wobei der de-facto Standard für Verschlüsselung \ac{RSA} ist. Opera unterstützt lediglich das überholte \ac{DHE}-Verfahren und Safari priorisiert Forward Secrecy niedrig und bevorzugt bei gegebener Option sogar die unverschlüsselte Kommunikation. Lediglich Firefox und Chrome unterstützen \ac{PFS} in vollem Umfang.
%Browser prüfen und mit eigenen Quellen versehen
%Dies ist die Vorlage für die einzelnen Kapitel, die jeweils mit Chapter als Kapiteltitel starten
\chapter{Transportwegverschlüsselung}
Das \ac{SSL}-Protokoll wurde zunächst durch die Firma Netscape entwickelt, um die Kommunikation über \ac{HTTP}-Verbindungen abzusichern.\footnote{Eckert, S. 796} \ac{SSL} kann auf der Sitzungs- und Präsentationsschicht des \ac{OSI}-Referenzmodells angesiedelt werden und setzt meist auf \ac{TCP} auf. Es hat die Aufgabe den darüber liegenden Schichten die Möglichkeit für eine authentifizierte, integritätsgeschützte und verschlüsselte Kommunikation zu geben.\footnote{Eckert, S. 799 ff.}
Die Version \ac{SSL} 3.0 hat sich mittlerweile als de facto Standard im Internet durchgesetzt und wird von allen gängigen Browsern unterstützt.\\
Das \ac{TLS}-Protokoll kann als Weiterentwicklung von \ac{SSL}  3.0 angesehen werden und liegt aktuell in der Version 1.2 vor. Da beide Protokolle in ihren Kernkonzepten übereinstimmen werden sie häufig synonym verwandt. Da \ac{TLS} jedoch eine Weiterentwicklung von \ac{SSL} ist, werden dort einige Erweiterungen eingeführt sowie unsichere Verfahren zur Berechnung von \ac{MAC}-Werten durch neuere Varianten ersetzt.\\
Beide Protokolle bestehen aus mehreren Schichten bzw. Unterprotokollen wobei das Record- und das Handshakeprotokoll von besonderer Bedeutung sind. Das Record-Protokoll ist für die Fragmentierung, Authentifizierung mittels \ac{MAC}und Verschlüsselung der zu übertragenden Daten zuständig. Mittels des Handshakeprotokolls werden Sitzungen zwischen den Kommunikationspartnern hergestellt.Dies bedeutet, dass die Kommunikationspartner authentifiziert werden und alle Informationen, die zur Berechnung des Shared Secret für die symmetrische Verschlüsselung der Daten benötigt werden, ausgetauscht werden. Die folgende Abbildung verdeutlicht den Ablauf eines solchen Sitzungsaufbaus. \footnote{An dieser Stelle soll ein MEssage Sequence Chart eingebaut werden, z.B. Sorge, S. 170}\\
	
Flaschenhals
%Literaturverzeichnis
\chapter*{Literaturverzeichnis}
\printbibliography[heading=offline,nottype=online]
\printbibliography[heading=online,type=online]
\end{document}