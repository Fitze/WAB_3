\chapter{Schlüsselaustausch}
%mit \section{title} wird ein Unterkapitel der ersten Gliederungsebene überschrieben
Für die sichere Kommunikation ist ein sicherer Schlüsselaustausch die Ausgangsbasis um Angriffe zu verhindern. ``Hierbei ist der geschützte Austausch geheimer Schlüssel symmetrischer Kryptosysteme von Interesse.''\footcite[S. 437]{Eckert2013} 
\section{Perfect Forward Secrecy - PFS}
%Problem
Beim klassischen Schlüsselaustausch werden die Sitzungsschlüssel durch den Public-Key innerhalb des Server-Zertifikat übertragen.\footcite[Vgl.][]{Boeck2013} Dies geschieht mittels \ac{RSA}-Verfahren. Verschlüsselte Kommunikation ist jedoch nur solang die Schlüssel geheim bleiben sicher. Die Gefahr beim klassischen \ac{RSA}-\ac{PKI}-Verfahren ist dass vergangene Kommunikation nachträglich zu jedem Zeitpunkt entschlüsselt werden kann, sobald Angreifer in Besitz des Private-Key sind.\medskip\\
%Idee
Sinnvoller ist es die Sitzungsschlüssel %REFERENZ Erklären was Sitzungsschlüssel sind
zum Einen nicht mehr zu übertragen und zum Anderen unabhängig voneinander ständig neu zu generieren und bei Terminierung zu löschen. Realisiert wird dies durch die Protokoll-Eigenschaft Forward Secrecy, die im Kryptographischen Fachjargon auch \ac{PFS}\footcite[Vgl.][]{Boeck2013} genannt wird. Die ``Anforderung an die Verfahren und Protokolle zur Schlüsselerneuerung besteht darin dafür zu sorgen, dass die Kenntnis eines Schlüssels, also dessen Aufdeckung, nicht dazu führt, dass damit auch vorherige und nachfolgende Schlüssel direkt aufdeckbar sind.''\footcite[S. 439]{Eckert2013} Wie notwendig \ac{PFS} geworden ist zeigen die jüngsten Ereignisse im Zusammenhang mit dem OpenSSL-Bug "Heartbleed", mit dem es sehr einfach war Private Schlüssel von Server auszulesen.\footcite[Vgl. ][]{Zhu2014} \medskip\\
%Umsetzung
Das \ac{DHE} Verfahren ermöglicht dabei als Basis die Aushandlung eines Sitzungsschlüssels bei dem die Kommunikationspartner verschiedene Nachrichten senden und sich auf einen Sitzungsschlüssel einigen können, ohne diesen je übertragen zu haben. Dieser Schlüssel ist auch nur für die aktuelle Verbindung gültig und wird anschließend gelöscht. Der Public-Key des Servers wird weiterhin übertragen, jedoch nur um den Schlüsselaustausch zu signieren. ``Abgeschlossene Sitzungen können somit im Nachhinein nicht mehr entschlüsselt werden.''\footcite[Vgl. ][]{Schulz2014} Die Verschlüsselungsverfahren \ac{TLS/SSL} und IPsec beherrschen bereits \ac{PFS}.\medskip\\
%Vorteile
Aufgezeichnete verschlüsselte Daten können somit bei Besitz des privaten Schlüssels nicht entschlüsselt werden. Zudem wird einfaches Belauschen einer aktiven Verbindung deutlich erschwert, denn es müsste die gesamte Kommunikation mit einem gezieltem \ac{MITM}-Angriff\ref{sec:mitm} manipuliert werden. Für diese Problematik gibt es wiederum moderne Ansätze wie DANE (Vgl. Kapitel \ref{sec:dane}), die in Kombination mit \ac{PFS} aktuell bei der Verschlüsselung von Verbindungen höchsten Sicherheitsansprüchen entsprechen, indem zusätzlich die Authentizität der Kommunikation gewährleistet wird.\medskip\\
%Nachteile
Nachteile gibt es lediglich bei der Verwendung des bereits überholten, und seit Jahren als geknackt bekannte \ac{DHE}-Verfahren, denn dabei verzögert sich zusätzlich der Verbindungsaufbau. Die Schlüssellänge ist Minimum 1024 Bit, und längere Schlüssel mit 2048 oder 4096 Bit sind dabei nicht sicherer.\footcite[Vgl.][]{Boeck2013} Der moderne Nachfolger mit elliptischen Kurven \ac{ECDH} gilt aktuell als sicher und benötigt dabei weniger als 1024 Bit und verzögert den Verbindungsaufbau nur unweigerlich.\medskip\\
%Grenzen
Obwohl es Forward Secrecy bereits seit 1999 im \ac{TLS} Standard 1.0\footcite[Vgl.][]{Boeck2013} vorgesehen ist und somit essenzieller Bestandteil von Verschlüsselung ist, hat sich \ac{PFS} noch nicht als Standard durchgesetzt.\footcite{SSLLabs} Dies liegt zum einen an den Webservern. Mit einem Apache Webserver ist nur eine Moduluslänge von 1024 Bit vorgesehen. Beim Einsatz von \ac{DHE} würden Provider damit ihre Server daher unsicher betreiben. Zum Anderen sind es auf Client-Seite die Browser die \ac{DHE} bzw. \ac{ECDH} lange Zeit ignoriert haben. Der Internet Explorer verschlüsselte nur nach DSS, wobei der de-facto Standard für Verschlüsselung bereits \ac{RSA} war. Opera unterstützte lediglich das überholte \ac{DHE}-Verfahren und Safari priorisiert Forward Secrecy niedrig und bevorzugt bei gegebener Option sogar die unverschlüsselte Kommunikation. Lediglich Firefox und Chrome unterstützen \ac{PFS} in vollem Umfang.\medskip\\
%Browser prüfen und mit eigenen Quellen versehen
Für die E-Mail Kommunikation ist \ac{PFS} essenziell für die Befriedigung hoher Sicherheitsbedürfnisse. Um die dazugehörigen Sicherheitsniveaus abzudecken müssen E-Mail-Server Forward Secrecy unterstützen. Im August 2013 war \ac{PFS} nur sporadisch verbreitet in der E-Mail-Kommunikationslandschaft.\footcite[Vgl. ][]{Schulz2014} Mittlerweile wird von vielen E-Mail-Providern auch \ac{PFS} angewandt. Die Umsetzung erfolgt jedoch noch zu zögerlich, wenn man die Tatsache betrachtet, dass \ac{PFS} im Zusammenhang mit Heartbleed der letzte Funken Hoffnung für betroffene Nutzer war, das zumindest vergangene Kommunikation nicht entschlüsselt werden kann.\footcite[Vgl. ][]{Zhu2014}
\section{OpenPGP - Open Pretty Good Privacy}
Ein anderer wichtiger Austausch von Schlüsseln, die für hohe Sicherheitsniveaus entscheidend sind, nämlich für die Ende-zu-Ende-Verschlüsselung, ist der mittels OpenPGP.
Die größte Schwierigkeit beim Austausch von Schlüsselinformationen für Ende-zu-Ende-Verschlüsselung ist der für Privatanwender unbequeme Weg die öffentlichen Schlüssel auszutauschen. Dies muss noch manuell erfolgen und ist daher noch nicht Standard heutiger E-Mail Kommunikation. Es gibt zwar die Möglichkeit private Schlüssel durch Dienste\footnote{iMessage von Apple für Instant Messaging} zu verteilen, jedoch ist diese Methode umstritten. Sie bietet zwar Ende-zu-Ende Verschlüsselung, jedoch ist der eigentliche Sinn des privaten Schlüssels verfehlt, wenn der Dienst diesen für den Nutzer bestimmt und auf den eigenen Server zwischenspeichert. Der Nutzer hat keinen Einfluss auf den privaten Schlüssel und ist damit nicht Eigentümer dieses Schlüssels, der in Folge dessen kein private Key ist. 
Selbst vereinzelte \ac{CA}'s gehen mit solcher Methodik vor\footcite{Kaps2014}, und untergraben damit die Sicherheit des ganzen \ac{CA}-Modell. \medskip\\
Für den Austausch von privaten Schlüsseln für Ende-zu-Ende Verschlüsselung ist heute OpenPGP eine gute Werkzeugumgebung für gängige E-Mail Clients, und die Verschlüsselung und den Schlüsselaustausch so bequem wie möglich zu gestalten.