\chapter{Schlüsselaustausch}
%mit \section{title} wird ein Unterkapitel der ersten Gliederungsebene überschrieben
\section{Perfect Forward Secrecy - PFS}
%Problem
Beim klassischen Schlüsselaustausch werden die Sitzungsschlüssel durch den Public Key innerhalb des Server-Zertifikat übertragen.\footnote{Golem} Dies geschieht mittels \ac{RSA}-Verfahren. Verschlüsselte Kommunikation ist jedoch nur solang die Schlüssel geheim bleiben sicher. Die Gefahr beim klassischen \ac{RSA}-\ac{PKI}-Verfahren ist dass vergangene Kommunikation nachträglich zu jedem Zeitpunkt entschlüsselt werden kann, sobald Angreifer in Besitz des Private Keys sind.\medskip\\
%Idee
Sinnvoller ist es die Sitzungsschlüssel %REFERENZ Erklären was Sitzungsschlüssel sind
zum Einen nicht mehr zu übertragen und zum Anderen unabhängig voneinander ständig neu zu generieren und bei Terminierung zu löschen. Realisiert wird dies durch die Protokoll-Eigenschaft Forward Secrecy, die im Kryptographischen Fachjargon auch \ac{PFS}\footcite[test]{Boeck2013} genannt wird. \medskip\\
%Umsetzung
Das \ac{DHE} ermöglicht die Aushandlung eines Sitzungsschlüssels bei dem die Kommunikationspartner verschiedene Nachrichten senden und sich auf einen Sitzungsschlüssel einigen können, ohne diesen je übertragen zu haben. Dieser Schlüssel ist auch nur für die aktuelle Verbindung gültig und wird anschließend gelöscht. Der Public-Key des Servers wird weiterhin übertragen, jedoch nur um den Schlüsselaustausch zu signieren. Die Verschlüsselungsverfahren \ac{TLS/SSL} und IPsec beherrschen bereits \ac{PFS}.\medskip\\
%Vorteile
Aufgezeichnete verschlüsselte Daten können somit bei Besitz des privaten Schlüssels nicht entschlüsselt werden. Zudem wird einfaches Belauschen einer aktiven Verbindung deutlich erschwert, denn es müsste die gesamte Kommunikation mit einem gezieltem \ac{MITM}-Angriff manipuliert werden. Für diese Problematik gibt es wiederum moderne Ansätze wie \ac{DANE}, die in Kombination mit \ac{PFS} aktuell bei der Verschlüsselung von Verbindungen höchsten Sicherheitsansprüchen entsprechen, indem zusätzlich die Authentizät der Kommunikation gewährleistet wird.\medskip\\ %REFERENZ MitM erklären 
%Nachteile
Nachteile gibt es lediglich bei der Verwendung des bereits überholten, und seit Jahren als geknackt bekannte \ac{DHE}-Verfahren, denn dabei verzögert sich zusätzlich der Verbindungaufbau. Die Moduluslänge der Schlüssel ist Minimum 1024 Bit, und längere Schlüssel mit 2048 oder 4096 Bit sind debi nicht sicher. Der moderne Nachfolger mit elliptischen Kurven \ac{ECDH} gilt aktuell als sicher und benötigt dabei weniger als 1024 Bit und verzögert den Verbindungsaufbau nur unweigerlich.\medskip\\
%Grenzen
Obwohl es Forward Secrecy bereits seit 1999 im \ac{TLS} Standard 1.0 \footnote{golem} vorgesehen ist und somit essenzieller Bestandteil von Verschlüsselung ist, hat sich \ac{PFS} noch nicht als Standard durchgesetzt.\footnote{https://www.trustworthyinternet.org/ssl-pulse/} Dies liegt zum einen an den Webservern. Mit einem Apache Webserver ist nur eine Moduluslänge von 1024 Bit vorgesehen. Beim Einsatz von \ac{DHE} würden Provider damit ihre Server daher unsicher betreiben. Zum Anderen sind es auf Client-Seite die Browser die noch nicht mitspielen. Der Internet Explorer verschlüsselt nur nach DSS, wobei der de-facto Standard für Verschlüsselung \ac{RSA} ist. Opera unterstützt lediglich das überholte \ac{DHE}-Verfahren und Safari priorisiert Forward Secrecy niedrig und bevorzugt bei gegebener Option sogar die unverschlüsselte Kommunikation. Lediglich Firefox und Chrome unterstützen \ac{PFS} in vollem Umfang.
%Browser prüfen und mit eigenen Quellen versehen