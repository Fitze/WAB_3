\usepackage[T1]{fontenc}
\usepackage[utf8]{inputenc}
\usepackage{mathptmx}				%Font: Times New Roman
%\usepackage[default]{droidserif}	%Font: Droid Serif
\usepackage[ngerman]{babel}
%Seitenränder einstellen
\usepackage[head=2cm,bottom=2cm, left=25mm, right= 25mm ]{geometry}
%Paket für Erzeugung eines Abkürzungsverzeichnis, über das auch per Befehl im späteren Text auf diese verwiesen werden kann
\usepackage[printonlyused]{acronym}
%Paket zur Gestaltung der Kopf- und Fusszeile bei KOMA-Script
\usepackage{scrpage2}
\usepackage[hidelinks]{hyperref}
%Bibliograpie Packages / Settings
\usepackage{ragged2e} % Ermöglicht Flattersatz mit Silbentrennung
\usepackage[babel,german=guillemets]{csquotes}
% damit werden Zitate in französische Anführungszeichen gesetzt
\usepackage[%
backend=biber,% damit wird bestimmt, dass Sie mit der biber.exe arbeiten
bibencoding=utf8,% steht für die Kodierung der .bib-Datei
bibwarn=true,% damit werden ggf. enstandene Fehler ausgegeben
style=alphabetic,% hier wird die DIN 1505-T2 eingebunden
%firstinits=true% der Familienname des Authors wird an die erste Stelle
% gesetzt und anschließend der Vorname nur mit dem ersten
% Buchstaben abgekürzt ausgegeben
]{biblatex}
\usepackage[babel]{microtype} % bringt optischen Randausgleich und
% minimale Skalierung der Buchstaben
\setlength\bibitemsep{8pt} % Abstand zwischen 2 Einträgen im Verzeichnis
% nachfolgend wird der Kopf des Literaturverzeichnisses bestimmt
\defbibheading{online}{\subsection*{Online-Quellen}}
\defbibheading{offline}{\subsection*{Literatur}}
\ExecuteBibliographyOptions{
%isbn=false, % falls die ISBN hinterlegt ist wird diese ausgeblendet
}
\addbibresource{bibliothek.bib} % Einbindung der Literaturquelldatei
%Fortlaufende Fußnoten
\usepackage{chngcntr}
\counterwithout{footnote}{chapter}
%Paket zum Einbinden von Grafiken
\usepackage[pdftex]{graphicx}
\usepackage{wrapfig}
\usepackage{caption}


% Bibtexkey vor Quellenangabe (Jedoch nur bei stlye=alphabetic)
\DeclareFieldFormat{labelalpha}{\thefield{entrykey}}
\DeclareFieldFormat{extraalpha}{}

\usepackage{filecontents}
\usepackage{multicol}
