\usepackage[T1]{fontenc}
\usepackage[utf8]{inputenc}
\usepackage{mathptmx}				%Font: Times New Roman
%\usepackage[default]{droidserif}				%Font: Droid Serif
\usepackage[ngerman]{babel}
%Seitenränder einstellen
\usepackage[head=2cm,bottom=3cm, left=25mm, right= 25mm ]{geometry}
%Paket für Erzeugung eines Abkürzungsverzeichnis, über das auch per Befehl im späteren Text auf diese verwiesen werden kann
\usepackage[printonlyused]{acronym}
%Paket zur Gestaltung der Kopf- und Fusszeile bei KOMA-Script
%\usepackage{scrpage2}			Ausgeblendet, da fancy Kopf und Fußzeilen handled, und inkompatibel ist
\usepackage[hidelinks]{hyperref}
%Bibliograpie Packages / Settings
\usepackage{ragged2e} % Ermöglicht Flattersatz mit Silbentrennung
\usepackage[babel,german=guillemets]{csquotes}
% damit werden Zitate in französische Anführungszeichen gesetzt
\usepackage{amstext}          % für Klartext via \text{} in Formeln
\usepackage[%
	backend=biber,% damit wird bestimmt, dass Sie mit der biber.exe arbeiten
	bibencoding=utf8,% steht für die Kodierung der .bib-Datei
	bibwarn=true,% damit werden ggf. enstandene Fehler ausgegeben
	%style=alphabetic,% hier alphabetic einblenden.
	style=din,% hier wird die DIN 1505-T2 eingebunden
]{biblatex}
\usepackage[babel]{microtype} % bringt optischen Randausgleich und
% minimale Skalierung der Buchstaben
\setlength\bibitemsep{8pt} % Abstand zwischen 2 Einträgen im Verzeichnis
% nachfolgend wird der Kopf des Literaturverzeichnisses bestimmt
\defbibheading{online}{\subsection*{Online-Quellen}}
\defbibheading{offline}{\subsection*{Literatur}}
\ExecuteBibliographyOptions{
%isbn=false, % falls die ISBN hinterlegt ist wird diese ausgeblendet
}
\addbibresource{bibliothek.bib} % Einbindung der Literaturquelldatei
%Fortlaufende Fußnoten
\usepackage{chngcntr}
%\counterwithout{footnote}{chapter}
%Paket zum Einbinden von Grafiken
\usepackage[pdftex]{graphicx}
\usepackage{wrapfig}
\usepackage{caption}


% Bibtexkey vor Quellenangabe
\DeclareFieldFormat{labelalpha}{\thefield{entrykey}}
\DeclareFieldFormat{extraalpha}{}

\usepackage{filecontents}
\usepackage{multicol}
\usepackage{relsize}

%Abstand Chapters einstellen
\renewcommand*{\chapterheadstartvskip}{\vspace*{-2.3\baselineskip}}% +2.3 ist Standard

\usepackage{array}
\usepackage{tabularx}			%Tabellen Handler
\usepackage{longtable}
\usepackage{booktabs}
%wird bereits oben importiert bei beides
%\usepackage{wrapfig} 			%Textfluss, damit bsp. Bilder den Textumfließen
%\usepackage{graphicx}			
\usepackage{prettyref}			%Automatische Referenzierung mit den hier definierten Formaten 
%%%%%%%%%%%%%%%%%%%%%%%%%%% Definition der Referenzierungsformate %%%%%%%%%%%%%%%%%%%%%%%%%%%%
%%% Für Abschnitte %%%
\newrefformat{sec}{s. Kap.~\ref{#1} \glqq\nameref{#1}\grqq , S. \pageref{#1}}
%%% Für SubAbschnitte %%%
\newrefformat{subsec}{s. Kap.~\ref{#1} \glqq\nameref{#1}\grqq , S. \pageref{#1}}
%%% Für SubSubAbschnitte %%%
\newrefformat{subsubsec}{s. Kap.~\ref{#1} \glqq\nameref{#1}\grqq , S. \pageref{#1}}
%%% Für Abbildungen %%%
\newrefformat{fig}{s. Abb.~\ref{#1} \glqq\nameref{#1}\grqq , S. \pageref{#1}}
%%% Für Tabellen %%%
\newrefformat{tab}{s. Tab.~\ref{#1} \glqq\nameref{#1}\grqq , S. \pageref{#1}} 
%%%%%%%%%%%%%%%%%%%%%%%%%%%%%%%%%%%%%%%%%%%%%%%%%%%%%%%%%%%%%%%%%%%%%%%%%%%%%%%%%%%%%%%%%%%%%%

\usepackage{xcolor}         	%Farbiger Text
\definecolor{darkred} 			%Definiert die Farbe mit der Kommentiert wird
{rgb}{0.7,0.0,0.0}

\usepackage{fancyhdr}			%Kontrolliert die Fuß- und Kopfzeilen
\usepackage{datetime}			%Damit das aktuelle Datum in der Fußzeile erzeugt wird.
		