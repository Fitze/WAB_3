%Dies ist die Vorlage für die einzelnen Kapitel, die jeweils mit Chapter als Kapiteltitel starten
\chapter{Titel des Kapitels}

%mit \section{title} wird ein Unterkapitel der ersten Gliederungsebene überschrieben
\section{Wichtiges Unterkapitel erster Gliederungsebene}
Lorem ipsum \dots

%mit \subsection{title} wird ein Unterkapitel der zweiten Gliederungsebene überschrieben
\subsection{Wichtiges Unterkapitel der zweiten Gliederungsebene}
Lorem ipsum \dots

%Vom Grundsatz war es das für die Erstellung von Kapiteln, jetzt kommt noch ein kleines Beispiel für Fußnoten
Dies ist ein großartiges Beispiel \footnote{Wenn einem nix einfällt muss man eben Quatsch schreiben} für eine Fußnote.