%Dies ist die Vorlage für die einzelnen Kapitel, die jeweils mit Chapter als Kapiteltitel starten
\chapter{Datensicherheit 1x1}
Unter Datensicherheit wird der Schutz von Daten in den Aspekten Verfügbarkeit, Vertraulichkeit und Integrität verstanden. \footnote(https://www.bsi.bund.de/DE/Themen/ITGrundschutz/ITGrundschutzKataloge/Inhalt/Glossar/glossar_node.html). Im Gegensatz dazu beschreibt Datenschutz den Schutz von und den vertrauensvollen Umgang mit persönlichen Daten. In der IT-Sicherheit wird zusätzlich des Aspekt der Authentizität berücksichtigt \footnote{http://www.datenschutz-berlin.de/content/technik/begriffsbestimmungen/verfuegbarkeit-integritaet-vertraulichkeit-authentizitaet}. In diesem Abschnitt werden diese Aspekte näher betrachtet, da diese für das Verstädnis der vorgestellten Techniken in den späteren Kapitel notwendig sind.

%mit \section{title} wird ein Unterkapitel der ersten Gliederungsebene überschrieben
\section{Verfügbarkeit}

Unter Verfügbarkeit wird das Vorhandensein von Infrastruktur, Software, sämtliche IT-Dienstleistungen sowie -Funktionalitäten und Daten verstanden, so dass die Anwender bei Bedarf darauf zugreifen und nutzen können. Um dies zu gewährleisten, muss verhindert werden, dass
\begin{itemize}
\item Daten verschwinden oder nicht zugreifbar sind, wenn sie gebraucht werden,
\item Programme nicht funktionsbereit sind, wenn sie aufgerufen werden sollen,
\item Hardware und sonstige notwendige Mittel nicht funktionsfähig oder gar verschwunden sind, wenn sie für die Verarbeitung benötigt wird. \footnote{http://www.datenschutz-berlin.de/content/technik/begriffsbestimmungen/verfuegbarkeit-integritaet-vertraulichkeit-authentizitaet}
\end{itemize}

\section{Integrität}
Unter der Integrität der Daten wird verstanden, dass die Daten nicht ohne Aufzufallen verändert werden können und somit vollständig übermittelt worden sind. Bei diesem Aspekt geht es demzufolge um die Unversehrtheit der Nachricht.

\section{Vertraulichkeit}

Unter Vertraulichkeit versteht man den Schutz der Nachricht vor unbefugtem Zugriff durch Dritte. Nur der gewünschte Empfänger soll in der Lage sein, den Inhalt der Nachricht zu erfahren. Dazu werden mathematische Verfahren genutzt, die in den späteren Kapiteln behandelt werden.

\section{Authentizität}

Bei der Authentizität geht es darum, nachzuweisen, dass die beteiligten Kommunikationspartner tatsächlich diejenigen sind, für die sie sich ausgeben.